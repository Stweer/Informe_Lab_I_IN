\section{CONCLUSIONES} 
\begin{enumerate}[1.]
	\item En un medio globalizado y audaz como el del mundo empresarial, podemos ver que el entorno en el que la inmensa mayor\'ia de las empresas tiene soportados los procesos de negocio con diferentes sistemas de informaci\'on y estrategias, los ubica en un mercado tan competitivo como el actual, Hoy se ha convertido en un problema, por lo que la Inteligencia de Negocios se rige como una pieza clave para ser proactivo a la hora de tomar mejores decisiones y de conseguir mejor control de negocio y ventajas que nos diferencien de la competencia.
\end{enumerate}

\begin{enumerate}[2.]
 	\item Lo que hemos aprendido es que la gran mayor\'ia de empresas no utilizan sistemas de inteligencia empresarial para gestionar sus negocios. Sin embargo,sabemos que si entienden el concepto, y saben que son herramientas muy enriquecedoras para la gesti\'on actual, a lo que añaden las siguientes ventajas para el  uso de Software de Inteligencia de Negocios:
\end{enumerate}

\begin{enumerate}[3.]
	\item En conclusi\'on, Business Analytics vs Business Intelligence tienen un inmenso potencial y existen muchos desaf\'ios presentes en estos dos sectores, especialmente relacionados con el campo de las redes t\'ecnicas y sociales. Las empresas deben recordar que las t\'ecnicas de an\'alisis de negocios no son las mismas que las de BI. Los requisitos de un campo son diferentes y tambi\'en lo son los beneficios de cada uno de ellos. Dicho esto, la tecnolog\'ia solo es efectiva si la empresa que invierte en ella puede usarla de manera adecuada y sistem\'atica. Al trabajar con los usuarios finales, los consultores pueden ayudar a las empresas a usar las herramientas adecuadas para que puedan usar los datos para tomar decisiones que empoderen a la marca y los lleven al siguiente nivel de crecimiento y desarrollo.
\end{enumerate}
